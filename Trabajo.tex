\documentclass[a4paper,titlepage]{article}

\usepackage[spanish,es-nodecimaldot]{babel}
\usepackage[utf8]{inputenc}
\usepackage[T1]{fontenc}

\usepackage[margin=0.9in]{geometry}
\usepackage{amsmath,amsfonts,graphicx,float,booktabs,url}
\usepackage[font=footnotesize,labelfont={sf,bf},textfont=sf]{caption}
\usepackage[usenames,dvipsnames]{xcolor}
\usepackage[plainpages=false,pdfpagelabels,hypertexnames=false,hidelinks]{hyperref}
\graphicspath{{./Img/}}

\makeatletter
\DeclareMathSizes{\@xpt}{\@xpt}{6}{5}
\makeatother

\title{\Huge\textbf{El problema de la cuantización de la interacción gravitatoria}}
\author{\textsf{Iyán Méndez Veiga}\\ \textsf{Víctor Rodríguez Bouza}}
\date{\texttt{TRG 2015-2016}}

\begin{document}
% Portada provisional. Luego ya la haremos más guapa si hay tiempo (que lo dudo)
\maketitle

\begin{abstract}\centering\textit{
Do not go gentle into that good night,\\
Old age should burn and rave at close of day;\\
Rage, rage against the dying of the light.}
\end{abstract}
%
\section{Introducción. ¿Por qué es necesaria la cuantización de la gravedad?} % Víctor
\subsection{Aproximación histórica.}
% No necesita explicación

\subsection{La necesidad de la gravedad cuántica.}
% Los problemas de la gravedad de Einstein y la base para la gravedad cuántica.

%
\section{Gravedad cuántica covariante.} % Tú o yo

%
\section{Aproximaciones canónicas.} % Tú o yo
\subsection{Quantum Geometrodynamics.}

\subsection{Gravedad cuántica de bucles (\textit{Loop Quantum Gravity}).}

%
\section{Teoría de cuerdas.}

%
\section{Teorías de campo efectistas.} % Víctor
% Con la bibliografía que tenemos del señor ese casi que la ponemos como una sección más...
%
\section{Otras aproximaciones.} % Tú o yo
% Lo que fuera aquí sería el complementario de lo anterior. Quizás podría escindirse según las opciones que hubiera...
%
\section{Conclusiones y perspectivas de futuro.} % ¿Los dos?

%
\section{Bibliografía}
\bibliographystyle{ieeetr}
\bibliography{Referencias}
\nocite{*}

\end{document}
